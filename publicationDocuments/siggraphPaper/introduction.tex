\section{Introduction}

In medicine as well as other areas, it is important to know the location of an object as it is moving across a surface.\\
**TALK ABOUT PLACES WHERE WE WANT TO KNOW THE LOCATION**\\
\\
**TALK ABOUT HOW WE WANT 3D VISUALIZATION THAT IS IN REAL TIME**\\
\\
**TALK ABOUT ELASTICITY OF SURFACE**\\
\\
**TALK ABOUT DISPLACEMENT AND ORIENTATION SENSORS**\\
\\
**TALK ABOUT DATA WE GET FROM SENSORS**\\
**MENTION THAT WE USE THE DATA TO GET 3D LOCATION**\\
\\
**MENTION CALIBRATION NECESSARY**\\
\\
**MENTION NEED FOR ALGORITHM TO FOLLOW SURFACE**\\
\\
**MENTION COMBINATION OF ALGORITHM TO FOLLOW SURFACE WITH ROTATION CALIBRATION**\\

\section{Old Introduction}

There are various medical device probes that take data at different points on the surface of a patient. It is very important to know the exact location of these points. One of the current methods used is attaching a grid to the patient and recording the data at each point in the grid. 
\\
This process is painstaking and we thus sought a way of automating it. We decided to use motion tracking technology to record the points faster. This has the added benefit of recording more points than previously. Because the surface of a patient is deformable, simple tracking in 3D would not suffice. We needed a method that would track directly on the surface. Because of that we use an optimal mouse sensor to track the displacement. Mouse sensors do not however tell you how you are oriented so we added a gyroscope, compass, and accelerometer to tell us that. \\
\\
In order to maximize the usefulness of this data, we wanted to show a 3D visualization of this path. We want to visualize this path in a 3D environment. As a reference then, we do a 3D scan of the patient beforehand and load it into our environment. There will be calibration points marked on the patient before the 3D scan and those will serve as starting points for our tracking. We now have the problem of given this 3D model and a starting point on it as well as the displacment and orientation data, how we do produce an accurate visualization of the path. Because we have a 3D environment, we need to convert the 2D displacement from the mouse sensor and rotation angles from the orientation equipment and output 3D points corresponding to the location of the probe on the virtual surface. \\
\\
Because we have a pre-loaded model to follow, we need to make sure that displacement in the real world is accurately shown as displacement in the virtual world. Additionally, we need to make sure that orientation in the real world matches the one we see in the virtual world. Once that happens, we can start producing 3D paths in the virtual world that resemble the path the probe took in the real world. Because the surface is deformable, we will also have to transform this path so that it directly follows the virtual surface. Once all of this is done we will be able to produce reasonably accurate paths in the virtual world that show the path the probe took in the real world. By coloring the different parts of the path depending on the data taken there, we can also give doctors an accurate picture of which parts of the surface of a patient are important.\\
\\
In this paper, in order to make this system work, we introduce a new method for tracking directly on a deformable surface. We also introduce a method of calibration for curved and deformable surfaces. This will help enable accurate visualizations in applications where accurate tracking on a surface is important.