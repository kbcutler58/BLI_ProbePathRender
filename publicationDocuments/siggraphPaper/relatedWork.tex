\section{Related Work}

\subsection{Location Tracking}

Various solutions have been proposed for tracking objects in 3D space. For the clinical setting, we needed a solution that was simple to implement, inexpensive, and provided millimeter level accuracy. We investigated a few systems that provide 3D locations, including RF location, motion tracking, motion capture, magnetic tracking, and Kinect tracking and found that they were all lacking in some way. \\
\\
Electromagnetic Tracking is the most popular system in medicine for tracking across a patient \ref{emTrackingReview}. We needed to track on the surface of a patient however, so it did not fit our needs **TODO: ADD MORE INFO**. \\
\\
The problem of tracking a sensor on the skin's surface was approached in \ref{3dUltrasound}. Much of the initial work on tracking the position of a probe using the orientatation and displacement was inspired by this method. In the paper however they do not deal with the deformability of the surface. This paper attempts to reconcile that problem. \\
\\
Real-time RF location systems seemed like an ideal solution due to the fact that they give a 3D location using the signal strength between the object and transmitters. Unfortunately it turned out they would not work for our needs due to the lack of accuracy. Chintalapudi et al \cite{ezlocation} described the EZ location by Microsoft as well as other location systems. As said in the conclusion of that paper, the most accurate system they described was accurate up to 70 cm. Since we need millimeter level accuracy this would not work for us. \\
\\
We could have tried using image-based motion tracking. This would involve putting calibration points on the probe and tracking the location of those points as the probe moves. There are many disadvantages to this technique hence we did not use it for our purposes. It does not have a guarantee of measurement accuracy, which is necessary for our case. Additionally, the fact that it is image based means the probe would always have to be in the view of the camera and in a clinical setting that can be hard to ensure. The post-processing of the images can be time consuming and in a clinical setting the results should come quickly. Finally, training the medical technicians taking the data to do all the steps for calibration then tracking that would be required is impractical. \\
\\
Using the Kinect for tracking would be better than just image-based tracking due to the fact that we have a depth map. The depth map would mean more accuracy and less processing time when trying to find the 3D location of the probe. Unfortunately, we would still need the probe to always be in view of the camera which may be impractical. Even though the accuracy is better, it is still not guaranteed in any way which is a problem for us. Making the Kinect system work in a clinical setting might also be impractical. Due to the disadvantages of the Kinect, we decided to investigate other methods for tracking. \\
\\
Motion Capture is a popular technique for film making that tracks a character's motion in 3D space. Using a motion capture system we would get 3D coordinates for the probe's position as it moves across a patient. Current motion capture systems however are expensive and cumbersome to set up. Both of these make it impractical for the clinical setting. **INSERT REFERENCE**\\
\\
There is work being done on using magnets for motion tracking \cite{magnetictracking}. While we could get measurement accuracy with this method and the setup would not be too difficult, the system would be cumbersome for patients. It would also only be limited to tracking across an area as big as the magnetic coils. The method we are proposing can be used on any mesh that is scanned in 3D. Additionally, we want to easily be able to track across the entire mesh that was scanned.\\
\\

\subsection{Deformable Surface Tracking}

Current work on deformable surface tracking is focused on tracking the deformations of the surface itself \cite{deformableobjecttracking,convexopt}. This paper focuses on deformable surfaces that deform when pressure is applied but then retain their shape afterwards. It is also focused on tracking objects on the surface rather than the surface itself.\\
\\

\subsection{Mesh Unfolding}

Being that we are tracking on a 2D surface embedded into 3D space, we thought about trying to flatten the mesh first and then track on it. We considered using a mesh flattening algorithm that used topological surgery \cite{meshunfolding}. Even though the mesh flattening does not have any holes or gaps when using this algorithm, there is no way to assure that our path will not leave the boundary of the flattened mesh. Additionally, we need an algorithm to work on arbitrary meshes and the mesh flattening was only proven to work with a handful of meshes. Due to these difficulties, we decided not to flatten the entire mesh and only do local flattening when we want the path to follow the mesh, which means we care about the triangle where the path is currently as well as its neighboring triangles. 