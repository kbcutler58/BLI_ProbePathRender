\section{Related Work}

\subsection{3D Tracking}

The displacement and orientation sensors give us a 3D location that is then reconciled with the surface. There are other technologies that exist that could give us a 3D location. These technologies all have significant disadvantages that make them impractical for our purposes. \\
\\
Motion capture that is used for movies or the Kinect system are examples of simple systems that provide 3D location. We could have also tried using multiple images to get the sense of depth and acquire a 3D location using that method. All of these ideas however require the object to always be in view of the camera. For the purposes of a clinical setting, that is not practical due to the fact that the operator's hand or something else could easily obstruct the view. Additionally, these methods do not provide very precise 3D locations and precision is quite important for our purposes. \\
\\
There is a major technology currently being researched called Electromagnetic Tracking \ref{emTrackingReview}. This would give a 3D location for a medical device probe. The systems currently in development do provide millimeter-level precision. These systems however have proven to be expensive and difficult to implement. Once our system has been perfected, it would be inexpensive and simple to operate.

\subsection{3D Ultrasound}

There have been other attempts to track a probe on the surface of a patient, most notably as a method to produce 3D ultrasounds \ref{3dUltrasound}. The ultrasound readings were taken continuously. The data from the orientation and displacement sensors was used to generate the path of the probe. The path data was used to align the ultrasound readings and ultimately generate a 3D data set of the inside of a patient. Much of the work in this paper on tracking the position of a probe using orientation and displacement was inspired by this paper. \\
\\
Our paper seeks to extend this work. The 3D ultrasound system only generated a path on the surface and did not reconcile the path with a 3D scan of the surface. They just used path information to help stitch together the data set. Our technology is meant to use this path information as well as a preloaded mesh in order visualize the path of the probe on the patient. \\
\\
The fact that the surface of a patient is deformable means the paths recorded will have to be reconciled with the surface. The 3D ultrasound paper did not consider the deformations of the surface of a patient. In our paper, we reconcile our recorded path with a 3D scan of the surface in order to modify our paths and make them more accurate. \\
\\
The 3D ultrasound paper also did a very simplified approach to the orientation of the sensor. We aim for more complete solutions. **INSERT DETAIL ON THIS** 

\subsection{Deformable Surface Tracking}

Current work on deformable surface tracking is focused on tracking the deformations of the surface itself \cite{deformableobjecttracking,convexopt}. Additionally these works focus on surfaces that do not retain their shape after deformation. For our purposes the surfaces are elastic and thus retain their shape. This paper is also focused on tracking objects on the surface rather than the surface itself.\\
\\

\subsection{Mesh Unfolding}

Since we are tracking on a 2D surface embedded in 3D space and using a 2D displacement sensor, we need to be able to do some flattening in order show accurately where the probe is currently located on the surface. There are mesh flattening algorithms which attempt to flatten an entire 3D model \cite{meshunfolding}. For our purposes however, flattening the entire model is not practical. Because this has to work on arbitrary surfaces, there is no guarantee that a flattening without holes or gaps will exist. Even if it does exist, there is no guarantee that the path will not travel beyond the boundary of the flattened mesh. Because of these shortfalls, we decided to only flatten a local neighborhood around the probe's current location. Due to the fact that we are only tracking on surfaces that can deform to accommodate a flat probe, we can guarantee that a local flattening will exist.